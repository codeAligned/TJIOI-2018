\documentclass[twoside]{article}
\usepackage[utf8]{inputenc}
\usepackage[margin=1in]{geometry}
\usepackage{titling}
\usepackage{afterpage}

\renewcommand\maketitlehooka{\null\mbox{}\vfill}
\renewcommand\maketitlehookd{\vfill\null}

\newcommand{\blank}{\vskip 3mm}
\setlength\parindent{0pt}
\renewcommand\thesection{\Alph{section}}

\newcommand\blankpage{%
    \null
    \thispagestyle{empty}%
    \addtocounter{page}{-1}%
    \newpage}

\begin{document}

\pagenumbering{gobble}

% Before the actual contest
\pagenumbering{roman}

% Begin the actual problems
\pagenumbering{arabic}

\section{Science Fair}

At the local science fair, students have to listen to a endless series of presentations.  There are are $N$ science fair competitors, numbered 0...$N-1$.  All the students who listened to presenter $a$ will move to presenter $a-1$ for the next round.  However, presenter $0$ is handing out free food!  Instead of going to presenter $N-1$ like good students, everyone who heard presenter $0$ will try to shorten their wait for more food.  To make it subtle, one person will go to presenter one, one to presenter two, and so on.  If $N$ people listen to presenter $0$, one will low-key stick around and hear it a second time.  If more than $N$ people listen to presenter $0$, one will go to each presenter, including $0$, and there rest will go home.

Let us define a configuration as the number of people at each presentation.  From a starting configuration, the configuration of students will eventually repeat.  After the configuration repeats, it will stay in a cycle that infinitely repeats the same sequence of configurations.  Given the initial configuration of listeners, find the period of the cycle.  Also find the number of configurations, possibly including the initial state, that will never be repeated.  


\textbf{SHORT NAME:} \verb|MERRY|
\blank
\textbf{INPUT FORMAT:}\\
The first line of input contains $N$, the number of presenters $a$ ($1 \leq N \leq 150$).

The second line contains the number of people listening to each presenter ($1 \leq a_i \leq 150$), separated by spaces.


\blank
\textbf{OUTPUT FORMAT:}\\
First print the number of steps in the cycle, followed by a space, and then the number of configurations that are never repeated.

Note: You can assume that both answers will be less than 20,000.

\blank
\textbf{SAMPLE INPUT:}
\begin{verbatim}
4
1 5 2 9
\end{verbatim}

%4
%2 3 49 25
%6
%C 0 10          % 10 3 49 25
%F 1 2 5         % 3 49          -> -1
%F 0 3 5         % 10 3 49 25    -> 25
%C 3 5           % 10 3 49 5
%F 0 3 5         % 10 3 49 5     -> 10
%F 1 3 7         % 3 49 5        -> 49

\textbf{SAMPLE OUTPUT:}
\begin{verbatim}
2 4
\end{verbatim}

The sequence of configurations is:
\begin{verbatim}
1 5 2 9 
6 2 9 0
3 10 1 1
11 2 2 0
3 3 1 1
4 2 2 0
3 3 1 1
\end{verbatim}
\end{document}
