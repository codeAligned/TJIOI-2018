\documentclass[twoside]{article}
\usepackage[utf8]{inputenc}
\usepackage[margin=1in]{geometry}
\usepackage{titling}
\usepackage{afterpage}

\renewcommand\maketitlehooka{\null\mbox{}\vfill}
\renewcommand\maketitlehookd{\vfill\null}

\newcommand{\blank}{\vskip 3mm}
\setlength\parindent{0pt}
\renewcommand\thesection{\Alph{section}}

\newcommand\blankpage{%
    \null
    \thispagestyle{empty}%
    \addtocounter{page}{-1}%
    \newpage}

\begin{document}

\pagenumbering{gobble}

% Before the actual contest
\pagenumbering{roman}

% Begin the actual problems
\pagenumbering{arabic}

\section{Cafeteria Chaos}

As part of the recent rennovation, TJ got a new cafeteria.  This cafeteria offers $N$ $(1\leq N \leq 5000)$ food options.  On the first day, there was complete chaos in the cafeteria.  The designers had not realized that any student who eats the spicy curry will immediately have to find some milk.  After drinking the milk, those students will need to find some cookies.  It turns out that every food causes students to search for another food.  Chips, for example, just make students get more chips.  For every minute of lunch that passes, each student will move on the the next food they want.

\blank
TJ has a famously long luch period, $K$ minutes long $(0\leq K \leq 10^9)$.  For each food that students can begin lunch eating, find what food they will want at the end of luch.  Note that if lunch is zero minutes long, students will still want their first food.


\textbf{SHORT NAME:} \verb|CHAOS|
\blank
\textbf{INPUT FORMAT:}\\
The first line contains $N$ and $K$ seperated by a space.

The second line contains $N$ integers, separated by spaces, $(0\leq a_i < N)$.  This means that food $i$ makes students want food $a_i$.

Note: For four out of ten test cases $K$ will be less than 2000.

\blank
\textbf{OUTPUT FORMAT:}\\

Print what food students will end wanting for each food.  Print each answer, starting from food 0, on its own line.

\blank
\textbf{SAMPLE INPUT:}
\begin{verbatim}
5 3
1 0 1 4 2
\end{verbatim}

%4
%2 3 49 25
%6
%C 0 10          % 10 3 49 25
%F 1 2 5         % 3 49          -> -1
%F 0 3 5         % 10 3 49 25    -> 25
%C 3 5           % 10 3 49 5
%F 0 3 5         % 10 3 49 5     -> 10
%F 1 3 7         % 3 49 5        -> 49

\textbf{SAMPLE OUTPUT:}
\begin{verbatim}
0
1
0
2
1
\end{verbatim}

\end{document}
