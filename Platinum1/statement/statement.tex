\documentclass[twoside]{article}
\usepackage[utf8]{inputenc}
\usepackage[margin=1in]{geometry}
\usepackage{titling}
\usepackage{afterpage}

\renewcommand\maketitlehooka{\null\mbox{}\vfill}
\renewcommand\maketitlehookd{\vfill\null}

\newcommand{\blank}{\vskip 3mm}
\setlength\parindent{0pt}
\renewcommand\thesection{\Alph{section}}

\newcommand\blankpage{%
    \null
    \thispagestyle{empty}%
    \addtocounter{page}{-1}%
    \newpage}

\begin{document}

\pagenumbering{gobble}

% Before the actual contest
\pagenumbering{roman}

% Begin the actual problems
\pagenumbering{arabic}

\section{Platinum 1}

Given an integer array $a$ of length $N$ ($1 \leq N \leq 10^4$, $1 \leq a_i \leq 10^3$), perform $Q$ queries ($1 \leq Q \leq 10^4$).

Queries will can be in the following formats:
\begin{itemize}
    \item \verb|C| \textit{i} \textit{v}: change the number at index $i$ (0-indexed) to the value $v$ ($1 \leq v \leq 10^3$).
    \item \verb|F| \textit{i} \textit{j} \textit{v}:
    find the maximal number $k$ with index between $i$ and $j$ (inclusive) such that $gcd(k, v) \neq 1$. In other words, $k$ is the maximum in $a[i...j]$ such that $k$ shares a factor (other than 1) with $v$ ($1 \leq v \leq 10^3$). If there is no such number, output $-1$.
\end{itemize}

\blank
Note: There are $168$ primes less than $10^3$.
\blank
\textbf{SHORT NAME:} \verb|TBD|
\blank
\textbf{INPUT FORMAT:}\\
The first line of input contains $N$, the length of the array $a$ ($1 \leq N \leq 10^4$).

The second line contains the elements of array $a$ ($1 \leq a_i \leq 10^3$), separated by spaces.

The third line contains the number of queries $Q$ ($1 \leq Q \leq 10^4$).

The fourth through $3+Q$ lines (inclusive) contain each query. Queries are either of the format \verb|C| \textit{i} \textit{v} or \verb|F| \textit{i} \textit{j} \textit{v} ($0 \leq i, j \leq N-1$, $1 \leq v \leq 10^3$).

\blank
\textbf{OUTPUT FORMAT:}\\
For each query that starts with \verb|F|, output the maximal integer $k$ in $a[i...j]$ such that $gcd(k, v) \neq 1$, or $-1$ if $k$ does not exist.

\blank
\textbf{SAMPLE INPUT:}
\begin{verbatim}
4
2 3 49 25
6
C 0 10
F 1 2 5
F 0 3 5
C 3 5
F 0 3 5
F 1 3 7
\end{verbatim}

%4
%2 3 49 25
%6
%C 0 10          % 10 3 49 25
%F 1 2 5         % 3 49          -> -1
%F 0 3 5         % 10 3 49 25    -> 25
%C 3 5           % 10 3 49 5
%F 0 3 5         % 10 3 49 5     -> 10
%F 1 3 7         % 3 49 5        -> 49

\textbf{SAMPLE OUTPUT:}
\begin{verbatim}
-1
25
10
49
\end{verbatim}


\end{document}
